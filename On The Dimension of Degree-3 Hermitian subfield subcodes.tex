\documentclass[a4paper]{amsart}
\usepackage[utf8]{inputenc}
\usepackage{setspace}
\usepackage[margin=1.25in]{geometry}
\usepackage{graphicx}
\graphicspath{ {./figures/} }
%\usepackage{subcaption}
\usepackage{mathrsfs}
\usepackage{amsmath}
\usepackage{amssymb}
\usepackage{amsfonts}
\usepackage{amsthm}

%%%%%%%%%%%%%%%%%%%%%%%%%%%%%%%%%%
\usepackage{enumerate}
\usepackage{xcolor}
\usepackage{geometry}
\usepackage{hyperref} %Références cliquables
%\usepackage{pifont}
%\usepackage{caption}
%\usepackage{animate}
%\usepackage{verbatim}
%\usepackage{cleveref}
%\usepackage{csquotes}
%\usepackage{mathtools}
%\usepackage{relsize}
%\usepackage{tcolorbox}
%\usepackage{tikz}
%\usepackage{fontawesome5}
%\usepackage{academicons}
\usepackage{mdframed}
\mdfdefinestyle{algoBoxStyle}{%
	linecolor=black,
	linewidth=1pt,
	roundcorner=5pt,
	innertopmargin=10pt,
	innerbottommargin=10pt,
	innerrightmargin=10pt,
	innerleftmargin=10pt,
	backgroundcolor=white
}
%%%%%%%%%%%%%%%%%%%%%%%%%%%%%%%%%%
%$\usepackage{lineno}
%\linenumbers
\theoremstyle{plain}
\newtheorem{theorem}{Theorem}[section]
\newtheorem{lemma}[theorem]{Lemma}
\newtheorem{proposition}[theorem]{Proposition}
\newtheorem*{corollary}{Corollary}
\theoremstyle{definition}
\newtheorem{definition}{Definition}[section]
\newtheorem{conjecture}{Conjecture}[section]
\newtheorem{example}{Example}[section]
\newtheorem{problem}[definition]{Problem}
\theoremstyle{remark}
\newtheorem*{remark}{Remark}
\newtheorem*{note}{Note}
\newcommand{\GRS}{\mathbf{GRS}}
\newcommand{\Frob}{{\mathrm{Fr}_{q^2}}}
\newcommand{\g}{\mathfrak{g}}

%\DeclareMathOperator{\tr}{Tr}

\newcommand{\C}{C_L(D,sP_{\infty})|_{\mathbb{F}_q}}

\newcommand{\trf}[2]{\tr_{\mathbb{F}_{#1}/\mathbb{F}_{#2}}}
%%%%%%%%%%%%%%%%%%%%%%%%%%%%%%%%%%%%%%%%%%%%%%%%%%%%%%%%%%%%

\DeclareMathOperator{\ev}{ev}
\DeclareMathOperator{\trace}{Tr}

%------------Commands-----------%
\newcommand{\Fq}{\mathbb{F}_q}
\newcommand{\Fqsq}{\mathbb{F}_{q^2}}
\newcommand{\LsP}{\mathscr{L}(sP)}
\newcommand{\calA}{\mathcal{A}}
\newcommand{\calB}{\mathcal{B}}
\newcommand{\calP}{\mathcal{P}}
\newcommand{\calH}{\mathcal{H}}
\newcommand{\calL}{\mathcal{L}}
\newcommand{\calC}{\mathcal{C}}
\newcommand{\calD}{\mathcal{D}}
\newcommand{\calO}{\mathcal{O}}
\newcommand{\calR}{\mathcal{R}}
\newcommand{\calS}{\mathcal{S}}
\newcommand{\calT}{\mathcal{T}}
\newcommand{\calX}{\mathcal{X}}
\newcommand{\fqm}{\mathbb{F}_{q^m}}
\newcommand{\fq}{\mathbb{F}_{q}}
\newcommand{\fqo}{\mathbb{F}_{q_0^2}}
\newcommand{\F}{\mathbb{F}}
\newcommand{\Z}{\mathbb{Z}}
\newcommand{\PP}{\mathbb{P}}
\newcommand{\R}{\mathbb{R}}
\newcommand{\Trm}[1]{\trace_{\mathbb{F}_{q^m}/\fq}\!\left(#1\right)}
\newcommand{\Tr}[1]{\trace\!\left(#1\right)}
\newcommand{\set}[1]{\left\{#1\right\}}
\newcommand{\Floor}[1]{\left\lfloor #1 \right\rfloor}
\newcommand{\Span}[1]{\operatorname{Span}\left(#1\right)}
\newcommand{\LT}[1]{\operatorname{LT}\left(#1\right)}
\newcommand{\LM}[1]{\operatorname{LM}\left(#1\right)}
\newcommand{\Supp}{\operatorname{Supp}}
\newcommand{\Div}{\operatorname{Div}}
\newcommand{\ssag}[1]{\operatorname{\mathsf{SSAG}}_{q}\left(#1\right)}
%\newcommand{\GRS}{\operatorname{\mathsf{GRS}}}
\newcommand{\gen}{\mathfrak{g}}
\newcommand{\degab}[1]{\deg_{a,b}\left(#1\right)}
\newcommand\TODO[1]{\textcolor{red}{TO DO: #1}}
%%%%%% Bibliography %%%%%%
% Replace "sample" in the \addbibresource line below with the name of your .bib file.



\title{On the dimension of Hermitian subfield subcodes from higher degree place}
\author{Sabira El Khalfaoui, G\'abor Nagy, Jade Nardi}

%%%%%% Affiliations %%%%%%
%\affil[1]{}


%%%%%% Date %%%%%%
% Date is optional
\date{}

%%%%%% Spacing %%%%%%
% Use paragraph spacing of 1.5 or 2 (for double spacing, use command \doublespacing)
\onehalfspacing

\begin{document}



%%%%%% Abstract %%%%%%
\begin{abstract}

\end{abstract}

%%%%%% Main Text %%%%%%

\maketitle

\section{Introduction}

The advent of quantum computers poses significant threats to classical cryptographic schemes, necessitating the development of post-quantum cryptographic primitives that are resilient against quantum attacks. In this context, Algebraic-Geometry (AG) codes have gained considerable attention due to their excellent error-correcting capabilities and potential applications in secure communication and cryptographic protocols. Among various classes of AG codes, subfield subcodes stand out for their inherent resistance to structural attacks, making them prime candidates for deployment in post-quantum cryptography.


\textcolor{red}{Constructingg subfield subcodes, a process also known as restriction, is a simple yet effective technique in cryptography for hiding a code's structure. This is especially useful in the McEliece cryptosystem, where it's important that the code's structure isn't easily recognized. Subfield subcodes help meet this security need, making them a fundamental element in designing secure cryptographic systems.}


\textcolor{red}{	This paper, we investigate subfield subcodes of Hermitian codes from higher degree place, with a particular emphasis on determining their exact dimensions...  \\An important application of subfield subcodes of AG codes is in the McEliece cryptosystem, a public-key encryption scheme that has withstood for several years and is renowned for its security against quantum attacks. The security of the McEliece cryptosystem hinges on the hardness of decoding random linear codes. By using subfield subcodes of AG codes as the underlying codes, we can achieve a system that not only inherits the quantum-resistant properties of these codes but also benefits from their efficient decoding algorithms.\\
	In this paper ....}


\section{Algebraic Geometry (AG) codes \label{sec}}

\subsection*{Hermitian curves and their divisors}

For more details we refer the readers to \cite{stichtenoth2009algebraic, stepanov2012codes}.

%		The Hermitian curve $\mathscr{H}_q$ over the finite field $\mathbb{F}_{q^2}$ in affine coordinates, is described by the equation:
%	
%	\[
%	\mathscr{H}_q: Y^q + Y = X^{q+1}.
%	\]
%	
%	This curve is characterized by a genus $\mathfrak{g} = \frac{q(q-1)}{2}$, classifying it as a maximal curve. This classification is further substantiated by the property $\#\mathscr{H}_q(\mathbb{F}_{q^2}) = q^3 + 1$. Notably, $\mathscr{H}_q$ encompasses a singular point at infinity, which is denoted as $P_{\infty}$.
%	
%	The Frobenius automorphism, denoted by $\text{Fr}_{q^2}$, is defined through the mapping:
%	
%	\[
%	\text{Fr}_{q^2}: \overline{\mathbb{F}}_{q^2} \to \overline{\mathbb{F}}_{q^2}, \quad x \mapsto x^{q^2}.
%	\]
%	
%	This automorphism extends to the points on the curve $\mathscr{H}_q$ by applying $\text{Fr}_{q^2}$ to their coordinates. Consequently, a point $P$ on $\mathscr{H}_q$ is deemed $\mathbb{F}_{q^2}$-rational if and only if it satisfies the condition $P = \text{Fr}_{q^2}(P)$. Viewing $\mathscr{H}_q$ over the algebraic closure $\overline{\mathbb{F}}_{q^2}$, there is a one-to-one correspondence between the points of $\mathscr{H}_q$ and the places of the function field $\overline{\mathbb{F}}_{q^2}(\mathscr{H}_q)$.
%	
%	For a given divisor $D = \lambda_1P_1 + \cdots + \lambda_kP_k$, consisting of points $P_1, \ldots, P_k$ on $\mathscr{H}_q$ and respective integers $\lambda_1, \ldots, \lambda_k$, its Frobenius image is articulated as:
%	
%	\[
%	\text{Fr}_{q^2}(D) = \lambda_1 \text{Fr}_{q^2}(P_1) + \cdots + \lambda_k \text{Fr}_{q^2}(P_k).
%	\]
%	
%	A divisor $D$ is classified as $\mathbb{F}_{q^2}$-rational if it fulfills the condition $D = \text{Fr}_{q^2}(D)$. It is particularly noteworthy that if all points $P_1, \ldots, P_k$ belong to $\mathscr{H}_q(\mathbb{F}_{q^2})$, then $D$ inherently qualifies as $\mathbb{F}_{q^2}$-rational, though the reverse may not always be valid. The degree of $D$ is defined as $\deg(D) = \lambda_1 + \cdots + \lambda_k$, with the support of $D$ being the collection of points $P_i$ for which $\lambda_i \neq 0$.


The Hermitian curve, denoted as $\mathscr{H}_q$, over the finite field $\mathbb{F}_{q^2}$ in affine coordinates, is given by the equation:
\[
	\mathscr{H}_q: Y^q + Y = X^{q+1}.
\]
This curve has a genus $g = \frac{q(q-1)}{2}$, classifying it as a maximal curve because it achieves the maximum number of $\mathbb{F}_{q^2}$-rational points, which is $\#\mathscr{H}_q(\mathbb{F}_{q^2}) = q^3 + 1$. Additionally, $\mathscr{H}_q$ possesses a unique singular point at infinity, denoted $P_{\infty}$.

A divisor on $\mathscr{H}_q$ is a formal sum $D = n_1 Q_1 + \cdots + n_k Q_k$ where $n_1, \cdots, n_k$ are integers, and $Q_1, \cdots, Q_k$ are points on $\mathscr{H}_q$. The degree of the divisor $D$ is defined as $\deg(D) = \sum_{i=1}^k n_i$. The valuation of $D$ at a point $Q_i$ is $v_{Q_i}(D) = n_i$, and the support of $D$ is the set $\{ Q_i \mid n_i \neq 0 \}$.

The Frobenius automorphism, denoted as $\text{Fr}_{q^2}$, is defined over the algebraic closure $\overline{\mathbb{F}}_{q^2}$ and acts on elements by
\[
	\text{Fr}_{q^2}: \overline{\mathbb{F}}_{q^2} \to \overline{\mathbb{F}}_{q^2}, \quad x \mapsto x^{q^2}.
\]
It acts on points of $\mathscr{H}_q$ by applying to their coordinates. A point $Q$ on $\mathscr{H}_q$ is $\mathbb{F}_{q^2}$-rational if and only if it is fixed by $\text{Fr}_{q^2}(Q)$. In $\overline{\mathbb{F}}_{q^2}$, points on $\mathscr{H}_q$ correspond one-to-one with the places of the function field $\overline{\mathbb{F}}_{q^2}(\mathscr{H}_q)$.

For a divisor $D$, its Frobenius image is given by
\[
	\text{Fr}_{q^2}(D) = n_1 \text{Fr}_{q^2}(Q_1) + \cdots + n_k \text{Fr}_{q^2}(Q_k).
\]
$D$ is $\mathbb{F}_{q^2}$-rational if $D = \text{Fr}_{q^2}(D)$. Notably, if all points $Q_1, \ldots, Q_k$ are in $\mathscr{H}_q(\mathbb{F}_{q^2})$, then $D$ is inherently $\mathbb{F}_{q^2}$-rational.
%%%%%%%%%%%%%%%%%%%%%%%%%%%%%%%%%%%%%%%%%%%%

\subsection*{Riemann-Roch space}

For a non-zero function $g$ in the function field $\bar{\mathbb{F}}_{q^2}$ and a place $P$ ,$v_P(g)$ stands for the order of $g$ at $P$. If $v_P(g) > 0$ then $P$ is a zero of $g$, while if $v_P(g) < 0 $,then $P$ is a pole of $g$ with multiplicity  $-v_P(g)$. The principal divisor of a non-zero function $g$ is $(g)= \sum_{P}v_P(g) P$.

The \emph{Riemann--Roch space} associated with an  $\mathbb{F}_{q^2}$-rational divisor $G$ is the $\mathbb{F}_{q^2}$ vector space \[\mathscr{L}(G) := \set{g \in \mathbb{F}_{q^2}(\mathscr{H}_q) \mid (g) + G \geq 0} \cup {0},\] with dimension $\ell(G)$.


From  \cite[Riemann's Theorem~1.4.17]{stichtenoth2009algebraic}, we have \[\ell(G) \geq \deg(G) +1 - \mathfrak{g},\] with equality if $\deg(G) \geq 2\mathfrak{g}-1$.


In this work, our primary focus is on an $\mathbb{F}_{q^2}$-rational divisor $G$ of the form $ sP$ where $P$ is a degree $r$ place in $\mathbb{F}_{q^2}(\mathscr{H}_q)$ and $s$ is a positive integer. In the extended constant field of $\mathbb{F}_{q^2}(\mathscr{H}_q)$ with degree $r$, let $P_1, P_2, \cdots, P_r$ be the extensions of $P$. These points are degree-one places in $\mathbb{F}_{q^{2r}}(\mathscr{H}_q)$, and, by appropriately labeling the indices, $P_{i} = \text{Fr}^i_{q^2}(P_1)$, where indices are considered modulo $r$.




%%%%%%%%%%%%%%%%%%%%%%%%%%%%%%%%%%%%%%%%%%%%%%%%%%%%%%
\subsection*{Hermitian codes}
Here, we outline the construction of an AG code from the Hermitian curve


In algebraic coding theory, Hermitian codes stand out as a significant class of algebraic geometry (AG) codes, renowned for their distinctive properties. These codes are constructed from Hermitian curves defined over finite fields. These codes are typically viewed as functional AG codes, denoted by $C_{\mathcal{L}}(D, G)$. In this standard approach, the divisor $G$ is usually a multiple of a single place of degree one. The set $\mathcal{P}$, encompassing all rational points on $\mathscr{H}_q$, is listed as $\{Q_1, \ldots, Q_{n}\}$. This approach gives rise to a structure referred to as a one-point code. However, it is important to note that recent research in the field suggests that using a more varied selection for the divisor $G$ can result in the creation of better AG codes \cite{matthews2005one,korchmaros2013hermitian}.



Given a divisor $D=Q_1+Q_2+\cdots+Q_n$ where all $Q_i$ are distinct rational points, and an $\mathbb{F}_{q^2}$-rational divisor $G$ such that $\Supp(G) \cap \calP = \varnothing$. By numbering the points in $\calP$, we define an evaluation map $\ev_{\calP}$ such that $\ev_{\calP}(g) = (g(Q_1),\dots,g(Q_n))$ for $g \in \calL(G)$. 


The functional AG code associated with the divisor $G$ is
\[C_{\calL}(D,G) := \{(g(Q_1),g(Q_2),\cdots,g(Q_n)) \mid g \in \calL(G)\},\]



%	a $[n, k \geq \ell(G)]_{\mathbb{F}_{q^2}}$ code. If $ \deg (G) < n $, its dimension is $\ell(G)$ and its minimum distance $d$ is bounded by the \emph{designed distance} $d^*=n-\deg (G)$.

\begin{theorem}\cite[Theorem~2.2.2]{stichtenoth2009algebraic}
	$C_{\calL}(D, G)$ is an $[n, k, d]$ code with parameters
	\[
		k=\ell(G)-\ell(G-D) \quad \text { and } \quad d \geq n-\deg G .
	\]
\end{theorem}

The dual of an AG code can be described as a residue code (see \cite{stichtenoth2009algebraic} for more details), \emph{i.e.}
\[ C_{\calL}(D,G)^{\perp} = C_{\Omega}(D,G).\]


Moreover, the differential code $C_{\Omega}(D, G)$ is analogous to the functional code $C_{\mathcal{L}}(D, W+D-G)$, where $W$ represents a canonical divisor of $\overline{\mathbb{F}}_{q^2}(\mathscr{H}_q)$. Notably, they share identical dimensions and minimum distances; however, this correspondence does not preserve all crucial properties of the code.


\subsection*{Subfield Subcode and trace code}
For the efficient construction of codes over $\mathbb{F}_q$, one approach involves working with codes originally defined over an extension field, $\mathbb{F}_{q^m}$. When considering a code $\calC$ within $\mathbb{F}_{q^m}^n$, a subfield subcode of $\calC$ is its restriction to the field $\mathbb{F}_q$
This process, often employed in defining codes like BCH codes, Goppa codes, and alternant codes, plays a foundational role.

Let $q$ be a prime power, and $m$ a positive integer. Let $C$ denote a linear code of parameters $[n,k]$ defined over the finite field $\mathbb{F}_{q^m}$. The \emph{subfield subcode} of $C$ over $\mathbb{F}_q$, represented as $C|_{\mathbb{F}_q}$, is the set
\[
	C|_{\mathbb{F}_q} = C \cap \mathbb{F}_q^n,
\]
which consists of all codewords in $C$ that have their components in $\mathbb{F}_q$. 


The subfield subcode $C|_{\mathbb{F}_q}$ is a linear code over $\mathbb{F}_q$ with parameters $[n,k_0,d_0]$, satisfying the inequalities $d \leq d_0 \leq n$ and $n-k \leq n-k_0 \leq m(n-k)$. Moreover, a parity check matrix for $C$ over $\mathbb{F}_q$ provides up to $m(n-k)$ linearly independent parity check equations over $\mathbb{F}_q$ for the subfield subcode $C|_{\mathbb{F}_q}$.

Typically, the minimum distance $d_0$ of the subfield subcode exceeds that of the original code $C$.

Let $\trace_{\mathbb{F}_{q^m} / \mathbb{F}_q}$ denote the trace function from $\mathbb{F}_{q^m}$ down to $\mathbb{F}_q$, expressed as
\[
	\trace_{\mathbb{F}_{q^m} / \mathbb{F}_q}(x) = x + x^q + x^{q^2} + \ldots + x^{q^{m-1}}.
\]
For any vector $c = (c_1, c_2, \ldots, c_n) \in \mathbb{F}_q^n$, we define
\[
	\trace_{\mathbb{F}_{q^m} / \mathbb{F}_q}(c) = \left( \trace_{\mathbb{F}_{q^m} / \mathbb{F}_q}(c_1), \trace_{\mathbb{F}_{q^m} / \mathbb{F}_q}(c_2), \ldots, \trace_{\mathbb{F}_{q^m} / \mathbb{F}_q}(c_n) \right).
\]
Furthermore, for a linear code $C$ of length $n$ and dimension $k$ over $\mathbb{F}_{q^m}$, the code $\trace_{\mathbb{F}_{q^m} / \mathbb{F}_q}(C)$ is a linear code of length $n$ and dimension $k_1$ over $\mathbb{F}_q$.

A seminal result by Delsarte connects subfield subcodes with trace codes:

\begin{theorem}[\cite{Del75}]\label{delsarte}
	Let $C$ be a $[n,k]$ linear code over $\mathbb{F}_q$. Then the dual of the subfield subcode of $C$ is the trace code of the dual code of $C$, i.e., 
	\[
		(C|_{\mathbb{F}_q})^{\perp} = \trace_{\mathbb{F}_{q^m} / \mathbb{F}_q}(C^{\perp}).
	\]
\end{theorem}
Finding the exact dimension of a subfield subcode of a linear code is typically a hard problem. However, a basic estimation can be obtained by applying Delsarte's theorem {\cite{Del75}}:
\begin{equation}
	\dim C|_{\fq} \geq n - m(n-k).
\end{equation}



In Chapter 9 of Stichtenoth's work \cite{stichtenoth2009algebraic}, various results are presented on subfield subcodes and trace codes of AG codes. We will extend and adapt these results to the context of Hermitian codes in this section, focusing on some specific cases for detailed discussion.



Applying  Theorem~9.1.6 in \cite{stichtenoth2009algebraic} to Hermitian codes:

\begin{theorem}

	Consider the Hermitian codes
	\[\calC_{\calL}:= C_{\calL}(D,G) \text{ and } \calC_{\Omega}:=C_{\Omega}(D,G),\]
	where $D=Q_1+\ldots+Q_n$ (with pairwise distinct places $Q_1, \ldots, Q_n$ of degree one), and $G=sP$ where $P$ is a degree $r$ pace on $\mathscr{H}_q$ with $\operatorname{supp} D \cap \operatorname{supp} G=\emptyset$ and $\deg G<n$. Suppose that $G_1$ is a divisor of $\mathbb{F}_{q^2}(\mathscr{H}_q)$ satisfying
	
	\begin{align}
		G_1 \leq G \quad \text { and } \quad q \cdot G_1 \leq G . \label{eq:condi}
	\end{align}
	
	Then
	
	\begin{equation}\label{eq:esti1}
		\dim \trace_{\mathbb{F}_{q^2} / \mathbb{F}_q}\left(\calC_{\mathcal{L}}\right) \leq \begin{cases}m\left(\ell(G)-\ell\left(G_1\right)\right)+1 & \text { if } G_1 \geq 0, \\ m\left(\ell(G)-\ell\left(G_1\right)\right) & \text { if } G_1 \ngtr 0,\end{cases}
	\end{equation}
	and
	\begin{equation}\label{eq:esti2}
		\left.\dim C_{\Omega}\right|_{\mathbb{F}_q} \geq \begin{cases}n-1-m\left(\ell(G)-\ell\left(G_1\right)\right) & \text { if } G_1 \geq 0, \\ n-m\left(\ell(G)-\ell\left(G_1\right)\right) & \text { if } G_1 \ngtr 0 .\end{cases}
	\end{equation}
	
\end{theorem}

The biggest divisor $G_1$ that satisfies the condition \eqref{eq:condi} (with respect to the degree) is the following:
\[G_1= \left[ \frac{q(q-1)}{r}\right]P \quad \text{ and } \quad  G=q.G_1, \]
in \eqref{eq:esti1} and \eqref{eq:esti2} we can replace $\ell (G_1)$ and $\ell(G)$ by $\deg G_1$ and $\deg G$ since $\deg G_1= q(q-1)=2\mathfrak{g}$, which  follows immediately from the Riemann-Roch Theorem.
Moreover, we derive the following corollary from Theorem~9.1.6~\cite{stichtenoth2009algebraic}

\begin{corollary}\label{coroll}
	With the notation as above. Let $P$ be a place on $\mathscr{H}_q$ of degree $r$ such that:
	\[G_1= \left[ \frac{q(q-1)}{r}\right]P \quad \text{ and } \quad  G=q.G_1, \]
	then 
	\[\dim C_{\mathcal{L}}(D,G_1)_{\mid \mathbb{F}_q}= 1.\]
\end{corollary}

\begin{proof}
	Let $f$ be a function in $\mathscr{L}(G_1)$ such that $f(Q_i) \in \mathbb{F}_q$ for $i=1,\cdots,n$. Then $f^q -f \in \mathscr{L}(G)$ (since $\mathscr{L}(G_1)^q \subseteq \mathscr{L}(G)$), hence $f^q - f \in  \mathscr{L}(G-D)$ where
	\[\mathscr{L}(G-D) = \operatorname{Ker}\left(\operatorname{ev}_{\calP}\right)=\left\{x \in \mathscr{L}(G) \mid v_{P_i}(x)>0 \text { for } i=1, \ldots, n\right\} ,\]
	since we assumed that $\deg (G-D)<n$, it follows that $f^q -f =0$ which implies that $f \in \mathbb{F}_q$. Consequently $\dim C_{\mathcal{L}}(D,G_1)_{\mid \mathbb{F}_q}= 1$.
\end{proof}



%%%%%%%%%%%%%%%%%%%%%%%%%%%%%%%%%%%%%%%%%%%%%%%%%%%%%%%%%%%%%%%%%%



\section{The geometry of Hermitian degree 3 places} \label{sec:geometry}

In this section we collect useful facts on the degree 3 places of the Hermitian curve, their stabilizer subgroups, and Riemann-Roch spaces. 

\subsection{The Hermitian sesquilinear form} \label{ssec:h-form}
The Hermitian curve $\mathscr{H}_q$ has affine equation $X^{q+1}=Y+Y^q$. The Hermitian function field $\bar{\mathbb{F}}_{q^2}(\mathscr{H}_q)$ is generated by $x,y$ such that $x^{q+1}=y+y^q$ holds. The Frobenius field automorphism $\Frob:x\mapsto x^{q^2}$ of the algebraic closure $\bar{\mathbb{F}}_{q^2}$ incudes an action on rational functions, places, divisors and curve automorphisms. For this action, we keep using the notation $\Frob$ in the exponent: $P^\Frob$, $f^\Frob$, $D^\Frob$, etc. 

Let $K$ be a field extension of $\mathbb{F}_{q^2}$. An affine point is a pair $(a,b)\in K^2$. A projective point $(a:b:c)$ is a $1$-dimensional subspace $\{(at,bt,ct) \mid t\in K\}$ of $K^3$. If $c\neq 0$, then the projective point $(a:b:c)$ is identified with the affine point $(a/c,b/c)$. For $u=(u_1,u_2,u_3), v=(v_1,v_2,v_3) \in K^3$, we define the Hermitian form 
\begin{align*} %\label{eq:}
\langle u,v \rangle = u_1v_1^q-u_2v_3^q-u_3v_2^q.
\end{align*}
Clearly, $\langle u,v \rangle$ is additive in $u$ and $v$, $\langle \alpha u, \beta v\rangle =\alpha \beta^q \langle u,v \rangle$, and 
\begin{align*} %\label{eq:}
\langle u,v \rangle^q=\langle v^\Frob,u \rangle.
\end{align*}
The point $u$ is self-conjugate, if 
\begin{align*} %\label{eq:}
0=\langle u,u \rangle = u_1^{q+1}-u_2u_3^2-u_2^qu_3.
\end{align*}
This is the projective equation $X^{q+1}-YZ^q-Y^qZ=0$ of the Hermitian curve $\mathscr{H}_q$. 

Let $u=(u_1:u_2u_3)$ be a projective point. The polar line of $u$ has equation
\[u^\perp: \langle (X_1,X_2,X_3), u \rangle = u_1^qX_1-u_3X_2-u_2X_3=0.\]
If $u$ is on $\mathscr{H}_q$, then $u^\perp$ is the tangent line at $u$. More precisely, $u^\perp$ intersects $\mathscr{H}_q$ at $u$ and $u^\Frob$ with multiplicities $q$ and $1$, respectively. If $u$ is $\mathbb{F}_{q^2}$-rational, then $u=u^\Frob$, and the intersection multiplicity is $q+1$. 

\subsection{Unitary transformations and curve automorphism} \label{ssec:unitary}

Let $A$ be a $3\times 3$ matrix. The linear map $u\mapsto uA$ will be denoted by $A$ as well. If $A$ is invertible, then it induces a projective linear transformation, denoted by $\hat{A}:(u_1:u_2:u_3)\mapsto (u_1':u_2':u_3')=(u_1:u_2:u_3)^{\hat{A}}$, where
\begin{align*} %\label{eq:}
u_1' &= a_{11}u_1+a_{21}u_2+a_{31}u_3, \\
u_2' &= a_{12}u_1+a_{22}u_2+a_{32}u_3, \\
u_3' &= a_{13}u_1+a_{23}u_2+a_{33}u_3. 
\end{align*}
We use the same notation $\hat{A}:(X,Y) \mapsto (X',Y')=(X,Y)^{\hat{A}}$ for the partial affine map
\begin{align*} %\label{eq:}
(X,Y) \mapsto (X',Y') = \left(\frac{a_{11}X+a_{21}Y+a_{31}}{a_{13}X+a_{23}Y+a_{33}}, \frac{a_{12}X+a_{22}Y+a_{32}}{a_{13}X+a_{23}Y+a_{33}}\right).
\end{align*}
The action $f(X,Y) \mapsto f((X,Y)^{\hat{A}^{-1}})$ of $\hat{A}$ on rational functions will be denoted by $A^*$. If $f(X,Y)$ is a polynomial of total degree $n$, then
\[f^{A^*}(X,Y)=\frac{h(X,Y)}{(a_{13}X+a_{23}Y+a_{33})^n},\]
where $h(X,Y)$ is polynomial of degree $n$. The line $a_{13}X+a_{23}Y+a_{33}=0$ can be seen as the pre-image of  the line at infinity under $\hat{A}$. 

The linear transformation $A$ is unitary, if
\[\langle uA, vA \rangle = \langle u,v \rangle\]
holds for all $u,v$. Since $\langle .,. \rangle$ is non-degenerate, unitary transformations are invertible. Moreover, for all $u,v$ one has
\begin{align*}
\langle (v^\Frob)A, uA \rangle &= \langle v^\Frob,u \rangle \\
&= \langle u,v \rangle^q \\
&= \langle uA,vA \rangle^q \\
&= \langle (vA)^\Frob,uA \rangle. \\
\end{align*}
This implies $(v^\Frob)A = (vA)^\Frob$ for all $v$, that is, $A$ and $\Frob$ commute. This shows that unitary transformations are defined over $\mathbb{F}_{q^2}$. They form a group, which is denoted by $GU(3,q)$. A useful fact is that if $b_1,b_2,b_3$ is a basis and
\[\langle b_iA, b_jA \rangle = \langle b_i,b_j \rangle\]
for all $i,j\in \{1,2,3\}$, then $A$ is unitary. 

Let $A\in GU(3,q)$. If $(x,y)$ is a generic point of $\mathscr{H}_q$, then $(x',y')=(x,y)^{\hat{A}}$ satisfies 
\[(x')^{q+1}-y'-(y')^q=\langle x',y' \rangle = \langle x,y \rangle=0. \]
Hence, $(x',y')$ is a generic point of $\mathscr{H}_q$, and $A^*$ induces an automorphism of the function field $\bar{\mathbb{F}}_{q^2}(\mathscr{H}_q)$. If $A$ is defined over $\Frob$, then $A^*$ is an automorphism of $\mathbb{F}_{q^2}(\mathscr{H}_q)$. 



\subsection{Places of degree 3 and their lines} \label{ssec:places-lines}
Let $a_1,b_1 \in \mathbb{F}_{q^6}\setminus \mathbb{F}_{q^2}$ be scalars such that $a_1^{q+1}=b_1+b_1^q$. In other words, $(a_1,b_1)$ is an affine point of $\mathscr{H}_q:X^{q+1}=Y+Y^q$, defined over $\mathbb{F}_{q^6}$. Write $a_2=a_1^{q^2}$, $b_2=b_1^{q^2}$, $a_3=a_2^{q^2}$, $b_3=b_2^{q^2}$, and $p_i=(a_i,b_i,1)$. Then $p_{i+1}=p_i^\Frob$, $\langle p_i,p_i \rangle = 0$, and 
\begin{align*} %\label{eq:}
0=\langle p_i,p_i \rangle ^q = \langle p_i^\Frob,p_i \rangle = \langle p_{i+1},p_i \rangle
\end{align*}
hold for $i=1,2,3$, the indices taken modulo $3$. Since $\langle .,. \rangle$ is non-trivial, $\gamma_i = \langle p_i,p_{i+1} \rangle \in \mathbb{F}_{q^6}\setminus \{0\}$. More precisely,
\begin{align*} %\label{eq:}
\gamma_1^{q^3}=\langle p_1,p_{2} \rangle^{q^3} = \langle p_2^\Frob,p_{1} \rangle^{q^2} = \langle p_2^{\Frob^2},p_{1}^\Frob \rangle = \langle p_1,p_2 \rangle = \gamma_1,
\end{align*}
which shows $\gamma_i \in \mathbb{F}_{q^3}\setminus \{0\}$. Clearly, $\gamma_{i+1} = \gamma_i^{q^2}$ and $\gamma_{i+2} = \gamma_i^{q}$. By $\gamma_i\neq 0$, the vectors $p_1,p_2,p_3$ are linearly independent over $\mathbb{F}_{q^6}$. 

Let $K$ be a field containing $\mathbb{F}_{q^6}$. Since $p_1,p_2,p_3$ is a basis in $K^3$, any $u\in K^3$ can be written as
\begin{align*} %\label{eq:}
u=x_1p_1+x_2p_2+x_3p_3,
\end{align*}
with $x_i\in K$. Computing
\begin{align*}
\langle u, p_{i+1} \rangle = \langle x_1p_1+x_2p_2+x_3p_3, p_{i+1} \rangle = x_i \langle p_{i}, p_{i+1} \rangle,
\end{align*}
we obtain $x_i=\langle u,p_{i+1} \rangle / \gamma_i$. In the basis $p_1,p_2,p_3$, the Hermitian form has the shape
\begin{align*}
\langle u,v \rangle &= \langle x_1p_1+x_2p_2+x_3p_3,y_1p_1+y_2p_2+y_3p_3 \rangle \\
&= x_1y_2^q \langle p_1,p_2 \rangle + x_2y_3^q \langle p_2,p_3 \rangle + x_3y_1^q \langle p_3,p_1 \rangle\\
&=\gamma_1 x_1y_2^q+\gamma_1^{q^2} x_2y_3^q+\gamma_1^{q^4} x_3y_1^q.
\end{align*}
In this coordinate frame, the Hermitian curve has projective equation
\begin{align*} %\label{eq:}
\gamma_1 X_1X_2^q+\gamma_1^{q^2} X_2X_3^q+\gamma_1^{q^4} X_3X_1^q=0.
\end{align*}

Let $x,y$ be the generators of the function field $\bar{\mathbb{F}}_{q^2}(\mathscr{H}_q)$ such that $x^{q+1}=y+y^q$. Write
\begin{align*} %\label{eq:}
\ell_i=\langle (x,y,1), p_i \rangle = a_i^qx-y-b_i^q.
\end{align*}
Then
\begin{align*} %\label{eq:}
(x,y,1)=\frac{\ell_{2}}{\gamma_1} p_1+\frac{\ell_{3}}{\gamma_2} p_2+\frac{\ell_{1}}{\gamma_3} p_3
\end{align*}
and
\begin{align} \label{eq:ell-q-lin-dep}
0=x^{q+1}-y-y^q = \langle (x,y,1), (x,y,1) \rangle = \frac{\ell_1\ell_{2}^q}{\gamma_1^q} + \frac{\ell_2\ell_{3}^q}{\gamma_2^q} + \frac{\ell_3\ell_{1}^q}{\gamma_3^q}.
\end{align}
The Hermitian curve $\mathscr{H}_q$ is non-singular, the places of $\bar{\mathbb{F}}_{q^2}(\mathscr{H}_q)$ correspond to the projective points over the algebraic closure $\bar{\mathbb{F}}_{q^2}$. Let $P_i$ denote the place corresponding to $(a_i:b_i:1)$. $P_i$ is defined over $\mathbb{F}_{q^6}$, $P_{i+1}=P_i^\Frob$, and 
\begin{align*} %\label{eq:}
P=P_1+P_2+P_3
\end{align*}
is an $\mathbb{F}_{q^2}$-rational place of degree $3$. 

The line $a_i^qX-Y-b_i^q=0$ is tangent to $\mathscr{H}_q$ at $p_i$, the intersection multiplicities are $q$ and $1$ at $p_i$ and $p_{i+1}$, respectively. This implies that the zero divisor $(\ell_i)_0$ is $qP_i+P_{i+1}$, and the principal divisor of $\ell_i$ is
\begin{align} \label{eq:div-ell-i}
(\ell_i)=qP_i+P_{i+1}-(q+1)Q_\infty.
\end{align}



\subsection{The stabilizer of a degree 3 place} \label{ssec:stabilizer}
Let $\beta_1\in \mathbb{F}_{q^6}$ be an element such that $\beta_1^{q^3+1}=1$. Define $\beta_2=\beta_1^{q^2}$, $\beta_3=\beta_2^{q^2}$. Then 
\begin{align*} %\label{eq:}
\beta_i\beta_{i+1}^q=\beta_i^{q^3+1}=1.
\end{align*}
For $p'_i=\beta_i p_i$, this implies
\begin{align*} %\label{eq:}
\langle p_i',p_{i+1}' \rangle = \beta_i\beta_{i+1}^q \langle p_i,p_{i+1} \rangle = \langle p_i,p_{i+1} \rangle.
\end{align*}
Hence, for all $i,j\in \{1,2,3\}$, 
\begin{align*} %\label{eq:}
\langle p_i',p_{j}' \rangle = \langle p_i,p_{j} \rangle.
\end{align*}
This shows that we can extend the map $p_i\mapsto p_i'$ to a unitary linear map $B=B(\beta_1):u\mapsto u'$ in the following way. Write
\begin{align*} %\label{eq:}
u=x_1p_1+x_2p_2+x_3p_3,
\end{align*}
with $x_i=\langle u,p_{i+1} \rangle / \gamma_i$, and define
\begin{align} \label{eq:Bdef}
u'=x_1p_1'+x_2p_2'+x_3p_3' = x_1\beta_1 p_1+x_2\beta_2p_2+x_3\beta_3p_3.
\end{align}
The extension $B$ is a unique unitary transformation. As we have seen in Section \ref{ssec:unitary}, this implies that $B=B(\beta_1)$ is a well-defined element of the general unitary group $GU(3,q)$. The set 
\[\mathcal{B}=\{B(\beta_1) \mid \beta_1 \in \mathbb{F}_{q^6}, \; \beta_1^{q^3+1}=1\}\]
is a cyclic subgroup of $GU(3,q)$, whose order is $|\mathcal{B}|=q^3+1$. 

In the projective plane, $B$ induces a projective linear transformation $\hat{B}$. $\hat{B}$ is trivial if and only if $\beta_1=\beta_2=\beta_1^{q^2}$, that is, if and only if $\beta_i\in \mathbb{F}_{q^2}$. As $\gcd(q^3+1,q^2-1)=q+1$, $\hat{B}$ is trivial if and only if $\beta_1^{q+1}=1$. The set $\hat{\mathcal{B}} = \{\hat{B} \mid B\in \mathcal{B}\}$ is a cyclic group of unitary projective linear transformations, whose order is $|\hat{\mathcal{B}}|=q^2-q+1$. 

In a similary way, we fix the elements 
\[\delta_i=\gamma_i^\frac{q^3-q}{2}.\] %\in \mathbb{F}_{q^6}
Since $\gamma_1 \in \mathbb{F}_{q^3}$, $\delta_{i}\in \mathbb{F}_{q^3}$. Moreover,
\[\delta_i^{q^3+1}=\delta_i^2=\gamma_i^{q^3-q}=\gamma_i^{1-q}.\] 
As before, the map 
\[\Delta:p_i\mapsto p_i''=\delta_i p_{i-1}\]
preserves the Hermitian form:
\[\langle p_i'',p_{i+1}'' \rangle = \langle \delta_{i} p_{i-1}, \delta_{i+1} p_{i} \rangle = \delta_{i}^{q^3+1} \langle p_{i-1},p_{i} \rangle = \gamma_i^{1-q} \gamma_{i-1} =\gamma_i.\]
Hence, $\Delta$ extends to a unitary linear map, which commutes with $\Frob$ and normalizes $\mathcal{B}$. Indeed,
\[ p_{i}^{\Delta^{-1}B\Delta} = (\delta_{i+1}^{-1}p_{i+1})^{B\Delta} = (\delta_{i+1}^{-1}\beta_{i+1} p_{i+1})^\Delta = \beta_{i+1} p_i, \]
hence, $\Delta^{-1}B\Delta=B^{q^2}$. $\Delta^3$ maps $p_i$ to $\delta_1\delta_2\delta_3 p_i$, and
\[\delta_1\delta_2\delta_3=\delta_1^{1+q+q^2}=\left(\gamma_1^\frac{q^3-q}{2}\right)^{1+q+q^2}= \left(\gamma_1^{q^3-1}\right)^\frac{(q+1)q}{q}=1.\]
$\Delta$ has order $3$. 

As introduced in Section \ref{ssec:unitary}, the unitary transformation $B$ and $\Delta$ induce automorphisms $B^*$ and $\Delta^*$ of the function field. 
\begin{proposition}
The group $\mathcal{B}^* = \{B^* \mid B \in \mathcal{B}\}$ of curve automorphisms has order $q^2-q+1$, and $\Delta^*$ normalizes $\mathcal{B}^*$ by 
\[(\Delta^*)^{-1}B^*\Delta^*=(B^*)^{q^2}=(B^*)^{q-1}.\]
Both $\mathcal{B}^*$ and $\Delta^*$ stabilize the degree $3$ place $P$. \qed
\end{proposition}




\section{Riemann-Roch spaces associated with a degree $3$ place} \label{sec:riemann-roch}
In this section, we keep using the notation of the previous section: $P_i$ is a degree 1 place of $\mathbb{F}_{q^6}(\mathscr{H}_q)$, associated to the projective point $(a_i:b_i:1)$. $P_i^\Frob=P_{i+1}$; the index $i=1,2,3$ is always taken modulo $3$. $P=P_1+P_2+P_3$ is an $\mathbb{F}_{q^2}$-rational place of degree $3$ of $\mathbb{F}_{q^2}(\mathscr{H}_q)$. The generators $x,y$ of $\bar{\mathbb{F}_{q^2}}(\mathscr{H}_q)$ satisfy $x^{q+1}=y+y^q$. The rational function $\ell_i=a_i^qx-y-b_i^q$ is obtained from the tangent line of $\mathscr{H}_q$ at $P_i$. 

\subsection{Basis and decomposition of the Riemann-Roch space}
Let $s,u,v$ be positive integers such that $v\leq q$ and $s=u(q+1)-v$. Clearly, $u,v$ are uniquely defined by $s$. In \cite{korchmaros2013hermitian}, the Riemann-Roch space associated with the divisor $sP$ is given as
\[\mathscr{L}(sP) = \left\{ \frac{f}{(\ell_1\ell_2\ell_3)^u} \mid f \in \mathbb{F}_{q^2}[X,Y], \; \deg f \leq 3u, \; v_{P_i}(f) \geq v \right\} \cup \{0\}.\]
The Weierstrass semigroup $H(P)$ consists of the integers $s\geq 0$ such that the pole divisor $(f)_\infty=sP$ for some $f\in {\mathbb{F}_{q^2}}(\mathscr{H}_q)$. If $s\not\in H(P)$, then it is called a Weierstrass gap; the set of Weierstrass gaps is denoted by $G(P)$. By \cite[Theorem 3.1]{korchmaros2013hermitian}, we have
\[G(P) = \{u(q+1)-v \mid 0\leq v\leq q, \; 0<3u\leq v\}.\]
By the Weierstrass Gap Theorem \cite[Theorem 1.6.8]{stichtenoth2009algebraic}, $|G(P)|=\g$ for a place of degree $1$. In our case, $P$ has degree $3$ and the situation is slightly more complicated.

\begin{lemma} \label{lm:GP-size}
\[3|G(P)| = \begin{cases}
\g & \text{if $q\equiv 0,1 \pmod3$,} \\
\g-1 & \text{if $q\equiv 2 \pmod3$.} \\
\end{cases}\]
\end{lemma}
\begin{proof}
The lemma follows from
\begin{align*}
|G(P)| &= \sum_{1 \leq u \leq q/3} |\{3u,\ldots,q\}| \\
&= \sum_{i=1}^{\lfloor q/3 \rfloor} q+1-3u \\
&= \frac{\lfloor q/3 \rfloor(2q-1-3\lfloor q/3 \rfloor)}{2}. \qedhere
\end{align*}
\end{proof}

The following proposition gives an explicit basis for the Riemann-Roch space $\mathscr{L}(sP)$ over the extension field $\mathbb{F}_{q^6}$.
\begin{proposition} \label{pr:Uti-props}
Let $t,u,v$ be positive integers such that $v\leq q$ and $t=u(q+1)-v$. Define the rational functions
\[U_{t,i} = \ell_i^{2u-v}\ell_{i+1}^{v-u}\ell_{i+2}^{-u} = \left(\frac{\ell_i}{\ell_{i+2}}\right)^u\left(\frac{\ell_{i+1}}{\ell_i}\right)^{v-u}, \qquad i=1,2,3.\]
Define $U_{0,i}=1$ as the constant function for $i=1,2,3$. Then the following hold:
\begin{enumerate}[(i)]
\item $(U_{t,i})^\Frob = U_{t,i+1}$. 
\item The principal divisor of $U_{t,i}$ is 
\[(U_{t,i}) = -tP + \big((3u-v-1)q+(q-v)\big) P_i + \big(v(q-2)+3u\big) P_{i+1}.\]
In particular, if $3u\geq v+1$, then $(U_{t,i}) \geq -tP$. 
\item The elements $U_{t,i}$, $t\geq 0$, $i=1,2,3$ are linearly independent with the following exception: $q\equiv 2 \pmod{3}$, $t=(q^2-q+1)/3$, 
\begin{align} \label{eq:U-lin-dep}
\frac{U_{t,1}}{\gamma_{1}^q} + \frac{U_{t,2}}{\gamma_{2}^q} + \frac{U_{t,3}}{\gamma_{3}^q} =0.
\end{align}
\item The set
\[\mathcal{U}(s) = \{U_{t,i} \mid t\in H(P), \; t\leq s, \; i=1,2,3, \; (3t,i) \neq (q^2-q+1,3)\}\]
of rational functions is a basis of $\mathscr{L}(sP)$ over $\mathbb{F}_{q^6}$. 
\end{enumerate}
\end{proposition}
\begin{proof}
Notice first that $u,v$ are uniquely defined by $t$, hence $U_{t,i}$ is well-defined. (i) is trivial, and (ii) is straightforward from \eqref{eq:div-ell-i}. To show (iii), let us write a linear combination in the form
\begin{align} \label{eq:U-lin-comb}
\alpha_1 U_{t,1}+\alpha_2 U_{t,2} + \alpha_3 U_{t,3} = \sum_{r<t, i} \lambda_{r,i} U_{r,i}
\end{align}
such that $(\alpha_1,\alpha_2,\alpha_3)\neq (0,0,0)$. The right hand side has valuation at least $-t+1$ at $P_1,P_2,P_3$. If $t\neq (q^2-q+1)/3$ and $\alpha_i\neq 0$, then the right hand side has valuation $-t$ at $P_{i+2}$. Hence, $\alpha_i=0$ for all $i=1,2,3$, a contradiction. Assume $t=(q^2-q+1)/3$. Then
\[U_{t,i}=\frac{\ell_i\ell_{i+1}^q}{(\ell_1\ell_2\ell_3)^\frac{q+1}{3}},\]
and \eqref{eq:U-lin-dep} follows from \eqref{eq:ell-q-lin-dep}. We can use \eqref{eq:U-lin-dep} to eliminate $U_{t,3}$ from \eqref{eq:U-lin-comb}, that is, we may assume $\alpha_3=0$. Then again, the only term which has valuation $-t$ at $P_{i+2}$ is $\alpha_iU_{t,i}$ with $\alpha_i\neq 0$. Since the left and right hand sides of \eqref{eq:U-lin-comb} must have the same valuations at $P_1,P_3$, $\alpha_1=\alpha_2=0$ must hold, a contradiction. 

(iv) By (iii), $\mathcal{U}(s)$ consists of linearly independent elements. To show that it is a basis of $\mathscr{L}(sP)$, it suffices to show that $|\mathcal{U}(s)|=\dim(\mathscr{L}(sP)$ for $3s\geq 2\g-2$. On the one hand, in this case $\dim(\mathscr{L}(sP) = 3s+1-\g$. On the other hand, 
\[|\mathcal{U}(s)|=1+3(s-|G(P)|)-\varepsilon = 3s+1-(3|G(P)|+\varepsilon),\]
where $\varepsilon=0$ if $q\equiv0,1\pmod3$, and $\varepsilon=1$ if $q\equiv 2\pmod3$. By Lemma \ref{lm:GP-size}, $3|G(P)|+\varepsilon = \g$, and the claim follows. 
\end{proof}

It is useful to have a decomposition of $\mathscr{L}(sP)$ over $\mathbb{F}_{q^2}$.

\begin{proposition} \label{pr:Wt-props}
For $t\geq 0$ integer and $\alpha \in \mathbb{F}_{q^6}$ define the rational function 
\[W_{t,\alpha} = \alpha U_{t,1} + \alpha^{q^2} U_{t,2} + \alpha^{q^4} U_{t,3}\]
and the $\mathbb{F}_{q^2}$-linear space
\[\mathcal{W}_t = \{W_{t,\alpha} \mid \alpha \in \mathbb{F}_{q^6}\}.\]
For $t\in H(P)$, we have
\[\dim(\mathcal{W}_t) = \begin{cases}
1 & \text{if $t=0$,} \\
2 & \text{if $q\equiv 2\pmod3$ and $t=(q^2-q+1)/3$,} \\
3 & \text{otherwise.}
\end{cases}\]
The $\mathbb{F}_{q^2}$-rational Riemann-Roch space $\mathscr{L}(sP)$ has the direct sum decomposition
\[\mathscr{L}(sP) = \bigoplus_{t\in H(P), \, t\leq s} \mathcal{W}_t.\]
\end{proposition}
\begin{proof}
For $t\in H(P)$, $\mathcal{W}_t$ is the set of $\mathbb{F}_{q^2}$-rational functions in the space spanned by $U_{t,1}, U_{t,2}, U_{t,3}$. The claims follow from Proposition \ref{pr:Uti-props}. 
\end{proof}

\subsection{Invariant subspaces of $\mathscr{L}(sP)$}

TODO

\section{Hermitian codes of degree $3$ places and their duals}

TODO \textcolor{red}{set the parameters $G=sP$ and $\tilde{D}=D+P_{\infty}$}
\subsection{Functional Hermitian codes of degree $3$ places}

Let $P$ be a degree $3$ place on the Hermitian curve $\mathscr{H}_q$. Given a divisor $D=Q_1+Q_2+\cdots+Q_n$, representing the sum of $\mathbb{F}_{q^2}$-rational affine points on $\mathscr{H}_q$, where $n=q^3$. For a positive integer $s$, we define the degree $3$ place functional Hermitian code $C_{\mathcal{L}}(D,sP)$ as:

\[C_{\mathcal{L}}(D,sP):= \left\lbrace \left(g(Q_1),g(Q_2),\cdots,g(Q_n) \right)\,\, |\,\, g \in  \mathcal{L}(sP) \right\rbrace, \]

This code forms an $[n,k]$ AG code, where $k\geq 3s-\g+1$, achieving equality when $\lfloor\frac{ 2 \g-2}{3}\rfloor <s<n$. Moreover, the code has a minimum distance $d\geq d^*=q^3-3s$, where $d^*$ the designed minimum distance.

Additionally, another degree $3$ place functional Hermitian code associated with $sP$, denoted by $C_{\mathcal{L}}(D+P_{\infty},sP)$, is constructed by evaluating functions in $\mathcal{L}(sP)$ on all rational points $Q_1,Q_2,\cdots,Q_n$, and the point at infinity $P_{\infty}$, as follows:

\[C_{\mathcal{L}}(D+P_{\infty},sP):= \left\lbrace \left(g(Q_1),g(Q_2),\cdots,g(Q_n),g(P_{\infty}) \right)\,\, |\,\, g \in  \mathcal{L}(sP)\right\rbrace ,\]

It shares the same dimension $k$ as $C_{\mathcal{L}}(D,sP)$ but has a length of $n+1$. 


\subsection{Differential Hermitian codes of degree $3$ places}
Differential Hermitian codes of degree $3$ places are essential counterparts to functional codes on the Hermitian curve $\mathscr{H}_q$. The dual code $C_{\Omega}(D,sP)$ of $C_{\mathcal{L}}(D,sP)$ is referred to as the differential code. It constitutes an $[n, \ell(sP -D)-\ell(sP) +\deg D, d^{\perp}]$ code, where $d^{\perp} \leq \deg(sP) - (2\g -2)$, with $\deg(sP) - (2\g -2)$ being its designed distance.

In \cite[Proposition 2.2]{korchmaros2013hermitian}, the authors provide an explicit description of an equivalence between the codes $C_{\Omega}(D,sP)$ and $C_{\mathcal{L}}(D,(q^3+q^2-q-2)P_{\infty}-sP)$ constructed on $\mathscr{H}_q$. This equivalence is related to the equivalence $C_{\Omega}(D,sP)= C_{\mathcal{L}}(D,W + D - sP)$ given in Section \ref{sec}.

Furthermore, the dual code $C_{\Omega}(D+P_{\infty},sP)$ of $C_{\mathcal{L}}(D+P_{\infty},sP)$ is denoted as $C_{\Omega}(D+P_{\infty},sP)$, which is equivalent to $C_{\mathcal{L}}(D+P_{\infty},W + D+ P_{\infty} - sP)$ and shares the same parameters as $C_{\Omega}(D+P_{\infty},sP)$ except for the length, which is $n+1$.

\section{The dimension of Hermitian subfield subcodes from degree 3 place (main result)}

\subsection{Dual codes}

\textcolor{red}{
In our study, we make use of a polynomial denoted as $R(X,Y) = X \prod\limits_{\substack{c \in \mathbb{F}_{q^2} \\ c^q + c \neq 0}} (Y-c)$, where $c$ ranges over $\mathbb{F}_{q^2}$ with $c^q + c \neq 0$. This polynomial plays a crucial role in our investigation of differential codes arising from a degree 3 place on the Hermitian curve $H$ defined over $\mathbb{F}_{q^2}$. We utilize its properties to derive our results, which are discussed further in this work.
In the function field, we observe a fundamental relationship which is expressed in the following propos
\begin{proposition}
	In the function field, we have $x^{q}R(x, y) = y^{q^2} - y$ and $R(x, y) = x^{q^2} - x$.
\end{proposition} 
\begin{proof}
	contenu...
\end{proof}
This proposition highlights a key aspect of the relationship between lines and Hermitian curves, specifically regarding their tangential interactions and intersections. For more in-depth insights into this topic, readers are encouraged to consult \cite[Section~2]{korchmaros2013hermitian}.}


\subsection{Riemann-Roch basis associated with a degree 3 place}
The main result of this paper deals with an $\mathbb{F}_{q^2}$-rational divisor $G=sP$ where $P$ is a degree-3 place in $\mathbb{F}_{q^2}(\mathscr{H}_q)$ and $s$ is a positive integer. As stated above, in the extended constant field of $\mathbb{F}_{q^2}(\mathscr{H}_q)$ with degree $3$, let $P_1, P_2, P_3$ be the extensions of $P$. These points are degree-one places in $\mathbb{F}_{q^6}(\mathscr{H}_q)$, and, by appropriately labeling the indices, $P_{j+1} = \text{Fr}(P_j)$, where $\text{Fr}$ is the $q^2$-th power Frobenius map and indices are considered modulo $3$. Additionally, $P$ can be identified with the $\mathbb{F}_{q^2}$-rational divisor $P_1+P_2+P_3$ in $\mathbb{F}_{q^6}(\mathscr{H}_q)$. The Riemann-Roch space associated with the divisor $sP$ \cite{korchmaros2013hermitian} is defined as:

\begin{equation}\label{ba1}
	\mathscr{L}(sP) = \left\{ \frac{f}{(\ell_1\ell_2\ell_3)^u} \mid f \in \mathbb{F}_{q^2}[X,Y], \deg f \leq 3u, v_{P_i}(f) \geq v \right\} \cup \{0\},
\end{equation}


where $\ell_i=0$ represents the equation of the tangent line at $P_i$ on $\mathscr{H}_q$, and $s = u(q+1) - v$ with $0 \leq v \leq q$.
%%%%%%%%%%%%%%%%%%%%%%%%%%%%%

\textcolor{red}{
After establishing the Riemann Roch space as outlined in Equation \ref{ba1} we introduce a set of fundamentals, in Equation \ref{ba2}. In contrast to the basis this new framework provides insights into the characteristics of the space. It comprises components of the type $\left(\frac{\ell_i}{\ell_{i+2}}\right)^u\left(\frac{\ell_{i+1}}{\ell_i}\right)^{v. U}$, where $\ell_i$ denotes the equation of the line at $P_i$, on $\mathscr{H}_q$ as previously indicated. The parameters $u$ and  $v$ are bound by conditions.}

\begin{equation} \label{ba2}
	\mathscr{L}(s P)=\left\langle\left(\frac{\ell_i}{\ell_{i+2}}\right)^u\left(\frac{\ell_{i+1}}{l_i}\right)^{v - u} \left\lvert\, \begin{array}{l}
		0 \leqslant v \leqslant q \\
		v+1 \leqslant 3 u \\
		i=1,2,3
	\end{array} \quad(q+1) u-v \leqslant s\right.\right\rangle \\
\end{equation}

\begin{equation}\label{ba3}
	\mathscr{L}(s P) / \mathcal{L}((s-1) P)=\left\langle\left.\left(\frac{\ell_i}{\ell_{i+2}}\right)^u\left(\frac{\ell_{i+1}}{\ell_i}\right)^{v-u} \right\rvert\, \begin{array}{l}
		i=1,2,3 \\
		s=(q+1) u-v 
	\end{array}\right\rangle
\end{equation}

\begin{proposition}
	The basis \ref{ba2} consists of eigenvectors of $\alpha$.
\end{proposition}
%%%%%%%%%%%%%%%%%%%%%%%%%%
\section{The dimension of Hermitian subfield subcodes from degree 3 place (main result)}
\subsection{Hermitian codes and their subfield subcodes from degree 3 place}
Let $n=q^3$ and the divisor $D=Q_1+Q_2+\cdots+Q_n$ be the sum of $\mathbb{F}_{q^2}$-rational affine points of $\mathscr{H}_q$. For a positive integer $s$, we denote by $C_{\calL}(D,sP)$ the degree-3 place functional AG code. This has length $n=q^3$. If $2 \mathfrak{g}-2<s<n$, then the dimension of $C_{\calL}(D,sP)$ is $k=3s-\mathfrak{g}+1$ which is equal to the dimension of the Riemann-Roch space $\mathscr{L}\left(s P\right)$, and  the designed minimum distance of $C_{\calL}(D,sP)$ is $d=q^3-3s$.

\textcolor{red}{\begin{proposition}
	The generators of $C_{\mathcal{L}}\left(D+P_{\infty}, s P\right)$ can be expressed in terms of the basis \ref{ba2} for $s=2 \g$ and $s=2 \g+1$.
\end{proposition}}

In our study, we carried out experiments to accurately compute the exact dimension of the subfield subcodes $ C_{q}(s) $ for $ q \leq 16 $ and $0 \leq s \leq n$. Alongside these investigations of the dimension of the Hermitian code $ C_{\mathcal{L}}(\mathcal{P}, G) $ and its trace code, we noted an unusual behavior in the dimension when considering $s = q - 1 $, which leads to the following proposition:


\begin{proposition}

	Let $q\geq 3$, and $C_{\mathcal{L}}(D,G)$ be the Hermitian code associated with the divisor $G=(q-1)P$, where $P$ is a degree 3 place, then
	\[\dim C_{\mathcal{L}}(D,G)=4.\]
	
\end{proposition}

\begin{proof}
	\textcolor{red}{A voir (rewrite)}
	
	Let $\ell_i=0$ be the line $P_iP_{i+1}$, it is the tangent to $\mathscr{H}_q$ at $P_i$. More precisely, the intersection divisor of $\ell_i$ and $\mathscr{H}_q$ is $qP_i+P_{i+1}$. This implies that the principal divisor of $\ell_1/\ell_2$ satisfies
	\[\mathrm{div}(\ell_i/\ell_{i+1})=qP_i-(q-1)P_{i+1}-P_{i+2}.\]
	For $\alpha \in \mathbb{F}_{q^6}$, w define the function
	\begin{eqnarray*}
		w_\alpha&=&\alpha\ell_1/\ell_2+(\alpha\ell_1/\ell_2)^{\mathrm{Frob}_{q^2}}+(\alpha\ell_1/\ell_2)^{\mathrm{Frob}_{q^2}^2}\\
		&=&\alpha\ell_1/\ell_2+\alpha^{q^2}\ell_2/\ell_3+\alpha^{q^4}\ell_3/\ell_1.
	\end{eqnarray*}
	On the one hand, $w_\alpha$ is defined over $\mathbb{F}_{q^2}$. On the other hand, Korchm\'aros and Nagy showed in \cite[Theorem ~3.1]{korchmaros2013hermitian}
	\[v_{P_i}(w_\alpha)=-q+1.\]
	Hence, $w_\alpha$ is contained in the Riemann-Roch space $\mathcal{L}((q-1)P)$. In fact, $\dim\mathcal{L}((q-1)P)=4$ and $1,w_{\alpha_1}, w_{\alpha_2}, w_{\alpha_3}$ is a basis of $\mathcal{L}((q-1)P)$, provided $\alpha_1,\alpha_2,\alpha_3$ is an $\mathbb{F}_{q^2}$-basis of $\mathbb{F}_{q^6}$. 
\end{proof}







\begin{conjecture} 
	For a prime power $q \geq 3$, let  $\calC_{\mathcal{L}}=\calC_{\mathcal{L}}( D, (q-1)P)$ be the Hermitian code, where $P$ is a degree 3 place. Let $Tr(\calC_{\mathcal{L}})$ denotes the trace code $\trace_{\mathbb{F}_{q^2} / \mathbb{F}_q}(\calC_{\mathcal{L}})$. We conjecture that:
	\[\dim Tr(\calC_{\mathcal{L}}) = 7 .\]
\end{conjecture}


Experimental results indicate that for $0 \leq s < 2\mathfrak{g}$, the dimension of $C_{\mathcal{L}}(D,sP)_{\mid \mathbb{F}_q}$ is $1$. Additionally, in the corollary \ref{coroll} presented earlier demonstrates this result for $ s =\frac{q(q-1)}{3}=\frac{2}{3} \g$.



\begin{theorem}
	For a prime power $q \geq 3$, let $C_q (s)=C_{\mathcal{L}}( \mathcal{P},G)_{|_{\mathbb{F}_q}}$ denote the subfield subcode of the degree-3 place one-point Hermitian code. Then 
	\[ \dim C_{q}(s)= \begin{cases} 1 & \text { for } 0 \leq s<2\mathfrak{g} \\ 7 & \text { for } s=2\mathfrak{g} \text{ and } q>2\\ 10 & \text{ for } s=2\g +1 \text{ and } q>3\end{cases}\]
	
	
\end{theorem}
\begin{proof}
	\textbf{Case 1: $\mathbf{0 \leq s <\frac{2}{3}\g}$} or \textcolor{red}{from corollary ...}\\
	Observe that constant polynomials belong to $\mathscr{L}(sP)$ for all non-negative $s$, ensuring that $\dim C_{q}(s)\geq 1$. To establish that $\dim C_{q}(s) = 1$ for $0 \leq s <\frac{2}{3}\g$, we fix an arbitrary integer $s$ in this range and consider a generic element $(c_1,\ldots,c_{q^3}) \in C_{q}(s)$. This corresponds to a function $g$ in $\mathscr{L}(sP)$ such that $c_i=g(Q_i)$ is an element of $\mathbb{F}_q$ for each $i=1,\ldots,q^3$. 
	
	Next, we note that there exists a $\gamma \in \mathbb{F}_q$ such that at least $q^2$ of the $c_i$ values are equal to $\gamma$. In other words, the function $g-\gamma$ is in $\mathscr{L}(sP)$ and has at least $q^2$ zeros on $\mathscr{H}_q$. However, a non-zero function in $\mathscr{L}(sP)$ cannot have more than $q(q-1)$ zeros, leading us to conclude that $g-\gamma$ must be the zero function. This implies that every $c_i$ is equal to $\gamma$, and hence, $C_{q}(s)$ consists of constant vectors. This completes the proof, demonstrating that $\dim C_{q}(s) = 1$ for $0 \leq s <\frac{2}{3}\g$.
	
	\textbf{Case 1 part 2: $s=2\mathfrak{g}-1$?}\\
	
	
	
	\textbf{Case 2: $s=2\mathfrak{g}$}\\
	
	
	
	Let $\ell_i=0$ be the line $P_iP_{i+1}$, it is the tangent to $\mathscr{H}_q$ at $P_i$. More precisely, the intersection divisor of $\ell_i$ and $\mathscr{H}_q$ is $qP_i+P_{i+1}$. This implies that the principal divisor of $\ell_1/\ell_2$ satisfies
	\[\mathrm{div}(\ell_i/\ell_{i+1})=qP_i-(q-1)P_{i+1}-P_{i+2}.\]
	For $\alpha \in \mathbb{F}_{q^6}$, w define the function
	\begin{eqnarray*}
		w_\alpha&=&\alpha\ell_1/\ell_2+(\alpha\ell_1/\ell_2)^{\mathrm{Frob}_{q^2}}+(\alpha\ell_1/\ell_2)^{\mathrm{Frob}_{q^2}^2}\\
		&=&\alpha\ell_1/\ell_2+\alpha^{q^2}\ell_2/\ell_3+\alpha^{q^4}\ell_3/\ell_1.
	\end{eqnarray*}
	On the one hand, $w_\alpha$ is defined over $\mathbb{F}_{q^2}$. On the other hand, Korchm\'aros and Nagy showed in [KN2013, Theorem 3.1]
	\[v_{P_i}(w_\alpha)=-q+1.\]
	Hence, $w_\alpha$ is contained in the Riemann-Roch space $\mathcal{L}((q-1)P)$. In fact, $\dim\mathcal{L}((q-1)P)=4$ and $1,w_{\alpha_1}, w_{\alpha_2}, w_{\alpha_3}$ is a basis of $\mathcal{L}((q-1)P)$, provided $\alpha_1,\alpha_2,\alpha_3$ is an $\mathbb{F}_{q^2}$-basis of $\mathbb{F}_{q^6}$. 
	
	This implies
	\[w_\alpha^q\in \mathcal{L}(q(q-1)P),\]
	and for all $\beta \in \mathbb{F}_{q^2}$, 
	\[W_{\alpha,\beta}=\beta w_\alpha +(\beta w_\alpha)^q\in \mathcal{L}(q(q-1)P).\]
	The following claims are straightforward to show:
	\begin{enumerate}
		\item For any $\mathbb{F}_{q^2}$-rational affine place $Q_i$, $W_{\alpha,\beta}(Q_i)\in \mathbb{F}_q$.
		\item $\mathcal{W}=\{W_{\alpha,\beta} \mid \alpha \in \mathbb{F}_{q^6}, \beta \in \mathbb{F}_{q^2}\}$ is a linear space over $\mathbb{F}_q$.
		\item $\dim_{\mathbb{F}_q}\mathcal{W}=6$ and $\dim_{\mathbb{F}_q}(\mathbb{F}_q+ \mathcal{W})=7$.
		\item $\mathrm{eval}_D(\mathbb{F}_q+ \mathcal{W})$ is a subspace of $C_{q(q-1)}$ of dimension $7$.
	\end{enumerate}
	This finishes the proof.
	
	
\end{proof}


\textcolor{red}{special case $q=2$ the dimension is 5 }



\newpage
\bibliography{sample}
\bibliographystyle{alpha}
\end{document}
